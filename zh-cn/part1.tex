% Need to compile under XeTex or XeLaTeX

\documentclass{beamer}

\input{preamble.tex}

\subtitle{第一节:基本操作}

\begin{document}

%%%%%%%%%%%%%%%%%%%%%%%%%%%%%%%%%%%%%%%%%%%%%%%%%%%%%%%%%%%%%%%%%%%%%%%%%%%%%%%
%%%%%%%%%%%%%%%%%%%%%%%%%%%%%%%%%%%%%%%%%%%%%%%%%%%%%%%%%%%%%%%%%%%%%%%%%%%%%%%
%%%%%%%%%%%%%%%%%%%%%%%%%%%%%%%%%%%%%%%%%%%%%%%%%%%%%%%%%%%%%%%%%%%%%%%%%%%%%%%
\begin{frame}
\titlepage
\end{frame}

%%%%%%%%%%%%%%%%%%%%%%%%%%%%%%%%%%%%%%%%%%%%%%%%%%%%%%%%%%%%%%%%%%%%%%%%%%%%%%%
%%%%%%%%%%%%%%%%%%%%%%%%%%%%%%%%%%%%%%%%%%%%%%%%%%%%%%%%%%%%%%%%%%%%%%%%%%%%%%%
%%%%%%%%%%%%%%%%%%%%%%%%%%%%%%%%%%%%%%%%%%%%%%%%%%%%%%%%%%%%%%%%%%%%%%%%%%%%%%%
\begin{frame}{为何学\LaTeX{}? Why \LaTeX{}?}
\begin{itemize}
\item 它的排版美观,结构清晰
\begin{itemize}
\item 特别是数学公式排版
\end{itemize}
%
\item 它是科学家自己设计的
\begin{itemize}
\item 庞大、活跃的用户群体
\end{itemize}
%
\item 它灵活,强大——各种文档都有对应宏包
\begin{itemize}
\item 论文,演讲,表格 \ldots 
\end{itemize}
\item 改变你编辑文档的思路
\begin{itemize}
\item \texttt{debug},文档结构 \ldots
\end{itemize}
\end{itemize}

\begin{block}{其他选择?}
\begin{itemize}
  \item \texttt{Word}可处理的图像类型有限;可编辑格式\texttt{.doc, .docx}文档需专门软件读取。
  \item \texttt{HTML}不能二维排版,\texttt{MathJax}基于\LaTeX 设计
\end{itemize}
\end{block}
\end{frame}

%%%%%%%%%%%%%%%%%%%%%%%%%%%%%%%%%%%%%%%%%%%%%%%%%%%%%%%%%%%%%%%%%%%%%%%%%%%%%%%
%%%%%%%%%%%%%%%%%%%%%%%%%%%%%%%%%%%%%%%%%%%%%%%%%%%%%%%%%%%%%%%%%%%%%%%%%%%%%%%
%%%%%%%%%%%%%%%%%%%%%%%%%%%%%%%%%%%%%%%%%%%%%%%%%%%%%%%%%%%%%%%%%%%%%%%%%%%%%%%
\begin{frame}[fragile]{工作模式?}
\begin{itemize}
\item 用纯文本 \bftt{plain text} 写代码,用命令\cmd{commands}控制排版.
\item \bftt{latex} 编译器会处理文本,生成排版过的文档。
\end{itemize}
\vskip 2ex
\begin{center}
\begin{minted}[frame=single]{latex}
你是\emph{萍,--凭,}--凭什么打我的儿子? 
\end{minted}
\vskip 2ex
\tikz\node[single arrow,fill=lime,font=\ttfamily\bfseries,%
  rotate=270,xshift=-1em]{\ \large{latex}\ \ \ };
\vskip 2ex
\fbox{你是\emph{萍,——凭,}——凭什么打我的儿子?}
\end{center}
\end{frame}

%%%%%%%%%%%%%%%%%%%%%%%%%%%%%%%%%%%%%%%%%%%%%%%%%%%%%%%%%%%%%%%%%%%%%%%%%%%%%%%
%%%%%%%%%%%%%%%%%%%%%%%%%%%%%%%%%%%%%%%%%%%%%%%%%%%%%%%%%%%%%%%%%%%%%%%%%%%%%%%
%%%%%%%%%%%%%%%%%%%%%%%%%%%%%%%%%%%%%%%%%%%%%%%%%%%%%%%%%%%%%%%%%%%%%%%%%%%%%%%
\begin{frame}[fragile]{更多例子-列表和图片}
\begin{exampletwoup}
\begin{itemize}
\item 奶茶
\item 擂茶
\item 酥油茶
\end{itemize}
\end{exampletwoup}
\vskip 2ex

\cmd{itemize}是模块名称,表示创建无序列表。

有序列表使用\cmd{enumerate}。

\vskip 2ex

\begin{exampletwoup}
\begin{figure}
\includegraphics{gerbil}
\end{figure}
\end{exampletwoup}

\tiny{Image license: \href{https://pixabay.com/en/animal-apple-attractive-beautiful-1239390/}{CC0}}
\end{frame}

\begin{frame}[fragile]{更多例子\ldots}
\myfont
\begin{exampletwoup}
\begin{equation}
\alpha + \beta + 1
\end{equation}
\end{exampletwoup}
\vskip 2em
\begin{exampletwoup}
{\CJKfontspec[FakeSlant = 0.2]
{STXIHEI.TTF}伪斜体(华文细黑)}
\end{exampletwoup}
\vskip 2em
\begin{exampletwoup}
\textit{italics表强调}
\end{exampletwoup}
\end{frame}

%%%%%%%%%%%%%%%%%%%%%%%%%%%%%%%%%%%%%%%%%%%%%%%%%%%%%%%%%%%%%%%%%%%%%%%%%%%%%%%
%%%%%%%%%%%%%%%%%%%%%%%%%%%%%%%%%%%%%%%%%%%%%%%%%%%%%%%%%%%%%%%%%%%%%%%%%%%%%%%
%%%%%%%%%%%%%%%%%%%%%%%%%%%%%%%%%%%%%%%%%%%%%%%%%%%%%%%%%%%%%%%%%%%%%%%%%%%%%%%
\begin{frame}[fragile]{改变编辑文本的态度}

\begin{itemize}
\item 使用命令描述生成方式,而非最终效果。
\item 按功能区分内容。
\item 整体考虑排版。
\end{itemize}
\end{frame}

%%%%%%%%%%%%%%%%%%%%%%%%%%%%%%%%%%%%%%%%%%%%%%%%%%%%%%%%%%%%%%%%%%%%%%%%%%%%%%%
%%%%%%%%%%%%%%%%%%%%%%%%%%%%%%%%%%%%%%%%%%%%%%%%%%%%%%%%%%%%%%%%%%%%%%%%%%%%%%%
%%%%%%%%%%%%%%%%%%%%%%%%%%%%%%%%%%%%%%%%%%%%%%%%%%%%%%%%%%%%%%%%%%%%%%%%%%%%%%%
\section{基本操作}

%%%%%%%%%%%%%%%%%%%%%%%%%%%%%%%%%%%%%%%%%%%%%%%%%%%%%%%%%%%%%%%%%%%%%%%%%%%%%%%
%%%%%%%%%%%%%%%%%%%%%%%%%%%%%%%%%%%%%%%%%%%%%%%%%%%%%%%%%%%%%%%%%%%%%%%%%%%%%%%
%%%%%%%%%%%%%%%%%%%%%%%%%%%%%%%%%%%%%%%%%%%%%%%%%%%%%%%%%%%%%%%%%%%%%%%%%%%%%%%
\subsection{入门}
\begin{frame}[fragile]{\insertsubsection}
\begin{itemize}
\item 最基本的 \LaTeX{} 文档:
\inputminted[frame=single]{latex}{basics.tex}
\item 所有命令都由\emph{backslash/反斜线} \keystrokebftt{\bs}开始。
\item 所有文档开头都是\cmdbs{documentclass} 命令。
\item 放在花括号里 \keystrokebftt{\{} \keystrokebftt{\}} 的命令 \emph{argument/参数} 告诉 \LaTeX{} 文档的类型:\bftt{article(普通文章)}.
\item \emph{comment/注释}前用百分号 \keystrokebftt{\%} 标明 --- \LaTeX{}
会忽略该行余下内容.
\end{itemize}
\end{frame}

\begin{frame}[fragile]{\insertsubsection}
\begin{itemize}
\item \LaTeX{} 中文输入:
\inputminted[frame=single]{latex}{basics-zh-cn.tex}
\item \emph{CKJutf8}用UTF8编码,\cmd{gbsn}是字体名称。
\item 使用中需将含有中文字符部分括起来
\item CJK亦可处理日韩文本
\end{itemize}
\end{frame}

\begin{frame}[fragile]{\insertsubsection}
\begin{itemize}
\item \LaTeX{} 其他中文处理方式:
\inputminted[frame=single]{latex}{basics-zh-cn1.tex}
\item 将分别设置中文字体和数学公式字体
\item 使用XeCJK宏包则必须用XeTeX编译
\end{itemize}
\end{frame}

%%%%%%%%%%%%%%%%%%%%%%%%%%%%%%%%%%%%%%%%%%%%%%%%%%%%%%%%%%%%%%%%%%%%%%%%%%%%%%%
%%%%%%%%%%%%%%%%%%%%%%%%%%%%%%%%%%%%%%%%%%%%%%%%%%%%%%%%%%%%%%%%%%%%%%%%%%%%%%%
%%%%%%%%%%%%%%%%%%%%%%%%%%%%%%%%%%%%%%%%%%%%%%%%%%%%%%%%%%%%%%%%%%%%%%%%%%%%%%%
\begin{frame}[fragile]{\insertsubsection{} \wllogo \small{平台}}
\begin{itemize}
\item Overleaf 提供\LaTeX 在线编译.
\vskip 2em
\begin{center}
\fbox{\href{\wlnewdoc{basics.tex}}{%
点击这里打开一个 \wllogo{}示例文档}}
\\[1ex]\scriptsize{}
为保证编译体验,最好使用 \href{http://www.google.com/chrome}{Google Chrome} 或新版 \href{http://www.mozilla.org/en-US/firefox/new/}{FireFox}浏览器。
\end{center}
\vskip 2ex
\item 在浏览幻灯页时,请在 Overleaf上自己试试示例代码。
\item \textbf{一定要动手!}
\end{itemize}
\end{frame}

%%%%%%%%%%%%%%%%%%%%%%%%%%%%%%%%%%%%%%%%%%%%%%%%%%%%%%%%%%%%%%%%%%%%%%%%%%%%%%%
%%%%%%%%%%%%%%%%%%%%%%%%%%%%%%%%%%%%%%%%%%%%%%%%%%%%%%%%%%%%%%%%%%%%%%%%%%%%%%%
%%%%%%%%%%%%%%%%%%%%%%%%%%%%%%%%%%%%%%%%%%%%%%%%%%%%%%%%%%%%%%%%%%%%%%%%%%%%%%%
\subsection{基本文本格式}
\begin{frame}[fragile]{\insertsubsection{}}
\small
\begin{itemize}
\item 将全文放在 \cmdbegin{document} 和 \cmdend{document}之间。
\item 通常直接输入即可。
\begin{exampletwouptiny}
英文单词间留空格即可:
star farming。

自然段以换两行表示。
\end{exampletwouptiny}
\item Space in the source file is collapsed in the output.
\begin{exampletwouptiny}
多个many      spaces空格。
中文间         空格不显示。

多个空格用
\emph{\char`\\ 反斜线}$+$空格
强制\ \ \ \ \ \ 转换。
\end{exampletwouptiny}
\end{itemize}
\end{frame}

%%%%%%%%%%%%%%%%%%%%%%%%%%%%%%%%%%%%%%%%%%%%%%%%%%%%%%%%%%%%%%%%%%%%%%%%%%%%%%%
%%%%%%%%%%%%%%%%%%%%%%%%%%%%%%%%%%%%%%%%%%%%%%%%%%%%%%%%%%%%%%%%%%%%%%%%%%%%%%%
%%%%%%%%%%%%%%%%%%%%%%%%%%%%%%%%%%%%%%%%%%%%%%%%%%%%%%%%%%%%%%%%%%%%%%%%%%%%%%%
\begin{frame}[fragile]{\insertsubsection{}:陷阱}
\small
\begin{itemize}
\item 前后引号:\\
用左上角的 \keystroke{\`{}} 键表左引号;单引号\keystroke{\'{}} 表右引号.
\begin{exampletwouptiny}
单引号: `文本'.

双引号: ``文本''.
\end{exampletwouptiny}

\item \LaTeX 中的保留符号:\\[1ex]
\begin{tabular}{cl}
\keystrokebftt{\%} & 百分号              \\
\keystrokebftt{\#} & 井号     \\
\keystrokebftt{\&} & ampersand                 \\
\keystrokebftt{\$} & 美元符号               \\
\end{tabular}
\item 直接输入会导致语法错误。在前面加上反斜线以\emph{强制转换}
\begin{exampletwoup}
\$\%\&\#!
\end{exampletwoup}
\end{itemize}
\end{frame}

%%%%%%%%%%%%%%%%%%%%%%%%%%%%%%%%%%%%%%%%%%%%%%%%%%%%%%%%%%%%%%%%%%%%%%%%%%%%%%%
%%%%%%%%%%%%%%%%%%%%%%%%%%%%%%%%%%%%%%%%%%%%%%%%%%%%%%%%%%%%%%%%%%%%%%%%%%%%%%%
%%%%%%%%%%%%%%%%%%%%%%%%%%%%%%%%%%%%%%%%%%%%%%%%%%%%%%%%%%%%%%%%%%%%%%%%%%%%%%%
\begin{frame}[fragile]{处理异常信息}
\begin{itemize}
\item \LaTeX{} 有可能会不理解你的命令。这时它会停止编译,并返回错误信息。
\item 比如,你若把 \cmdbs{emph} 拼成 \cmdbs{meph},或把 \cmdbs{main} 拼成 \cmdbs{mian},\LaTeX{} 就会出错,告诉你出现了\bftt{undefined control sequence(未定义命令)},因为``meph'' 和 ``mian''都不是可处理的命令。
\end{itemize}
\begin{block}{处理建议}
\begin{enumerate}
\item 别害怕!出错很正常。
\item 出现就要马上查错——如果是你刚刚输入的文本引起的,就从这里开始检查。
\item 如果有好几个,就从第一个开始——甚至可能是在它之前的文本出的错。
\end{enumerate}
\end{block}
\end{frame}

%%%%%%%%%%%%%%%%%%%%%%%%%%%%%%%%%%%%%%%%%%%%%%%%%%%%%%%%%%%%%%%%%%%%%%%%%%%%%%%
%%%%%%%%%%%%%%%%%%%%%%%%%%%%%%%%%%%%%%%%%%%%%%%%%%%%%%%%%%%%%%%%%%%%%%%%%%%%%%%
%%%%%%%%%%%%%%%%%%%%%%%%%%%%%%%%%%%%%%%%%%%%%%%%%%%%%%%%%%%%%%%%%%%%%%%%%%%%%%%
\begin{frame}[fragile]{排版练习 1}

\begin{block}{用 \LaTeX 处理以下文段:
\footnote{\url{http://en.wikipedia.org/wiki/Economy_of_the_United_States}}}

在2006年三月, 国会再次将国债限额 
上调\$0.79万亿 至\$8.97 万亿,
约占国民生产总值 68\%。 至2008年10月
4日, "2008年经济紧急稳定法案" 
已将目前国债上限上调至\$11.3 万亿。

按 1:6.8870 的汇率,\$11.3 万亿换算
成人民币约为\textyen77.82万亿。
\end{block}
\vskip 2ex
\begin{center}
\fbox{\href{\wlnewdoc{basics-exercise-1.tex}}{%
点击这里在 \wllogo{}中打开该文档}}
\end{center}

\begin{itemize}
\item 提示:注意特殊字符!
\item 试过之后,
\fbox{\href{\wlnewdoc{basics-exercise-1-solution.tex}}{%
这里有我的解答}}。
\end{itemize}
\end{frame}

%%%%%%%%%%%%%%%%%%%%%%%%%%%%%%%%%%%%%%%%%%%%%%%%%%%%%%%%%%%%%%%%%%%%%%%%%%%%%%%
%%%%%%%%%%%%%%%%%%%%%%%%%%%%%%%%%%%%%%%%%%%%%%%%%%%%%%%%%%%%%%%%%%%%%%%%%%%%%%%
%%%%%%%%%%%%%%%%%%%%%%%%%%%%%%%%%%%%%%%%%%%%%%%%%%%%%%%%%%%%%%%%%%%%%%%%%%%%%%%
\subsection{数学排版}
\begin{frame}[fragile]{\insertsubsection{}: \$ 符号}
\begin{itemize}
\item 美元符号 \keystrokebftt{\$}为何被预留?他们是用来标识文本中的公式的。\\[1ex]
\begin{exampletwouptiny}
% 纯文本:
令a和b为不等正整数,
再令 c = a - b + 1。

% 数学表达式经处理:
令$a$和$b$为不等正整数,
再令 $c = a - b + 1$。
\end{exampletwouptiny}
\item \$ 符号必须成对出现——一个标志【表达式】区域开始,另一个表示结束。
\item \LaTeX{} 自动处理字符间距;它会忽略你的空格。
\begin{exampletwouptiny}
令$y=mx+b$ 为 \ldots

令 $y = m x + b$ 为 \ldots
\end{exampletwouptiny}
\end{itemize}
\end{frame}

%%%%%%%%%%%%%%%%%%%%%%%%%%%%%%%%%%%%%%%%%%%%%%%%%%%%%%%%%%%%%%%%%%%%%%%%%%%%%%%
%%%%%%%%%%%%%%%%%%%%%%%%%%%%%%%%%%%%%%%%%%%%%%%%%%%%%%%%%%%%%%%%%%%%%%%%%%%%%%%
%%%%%%%%%%%%%%%%%%%%%%%%%%%%%%%%%%%%%%%%%%%%%%%%%%%%%%%%%%%%%%%%%%%%%%%%%%%%%%%
\begin{frame}[fragile]{\insertsubsection{}: 特殊记号}
\begin{itemize}
\item 上标前用折音号 \keystrokebftt{\^},
      下标前用下划线 \keystrokebftt{\_}。
\begin{exampletwouptiny}
$y = c_2 x^2 + c_1 x + c_0$
\end{exampletwouptiny}
\vskip 2ex

\item 如果上下标超过了一个字符,就需要用花括号\keystrokebftt{\{} \keystrokebftt{\}}括起来。
\begin{exampletwouptiny}
$F_n = F_n-1 + F_n-2$     % 嗯?

$F_n = F_{n-1} + F_{n-2}$ % 赞!
\end{exampletwouptiny}
\vskip 2ex

\item 也有输入希腊字母/其他符号的命令
\begin{exampletwouptiny}
$\mu = A e^{Q/RT}$

$\Omega = \sum_{k=1}^{n} \omega_k$
\end{exampletwouptiny}
\end{itemize}
\end{frame}

%%%%%%%%%%%%%%%%%%%%%%%%%%%%%%%%%%%%%%%%%%%%%%%%%%%%%%%%%%%%%%%%%%%%%%%%%%%%%%%
%%%%%%%%%%%%%%%%%%%%%%%%%%%%%%%%%%%%%%%%%%%%%%%%%%%%%%%%%%%%%%%%%%%%%%%%%%%%%%%
%%%%%%%%%%%%%%%%%%%%%%%%%%%%%%%%%%%%%%%%%%%%%%%%%%%%%%%%%%%%%%%%%%%%%%%%%%%%%%%
\begin{frame}[fragile]{\insertsubsection{}: 表达式}
\begin{itemize}
\item 如果公式繁杂, 就用
	\cmdbegin{equation} and \cmdend{equation}
	将它单行显示。
	所有符号自动转为\bftt{$\backslash$displaystyle}\\[2ex]
	\begin{exampletwouptiny}
The roots of a quadratic equation
are given by
\begin{equation}
x = \frac{-b \pm \sqrt{b^2 - 4ac}}
         {2a}
\end{equation}
where $a$, $b$ and $c$ are \ldots
	\end{exampletwouptiny}
\vskip 1em
{\scriptsize 注意: \LaTeX{} 会忽略大多数空格,但\bftt{equation}模式中不能有空行。不要像分段那样连换两行。}
\end{itemize}
\end{frame}

%%%%%%%%%%%%%%%%%%%%%%%%%%%%%%%%%%%%%%%%%%%%%%%%%%%%%%%%%%%%%%%%%%%%%%%%%%%%%%%
%%%%%%%%%%%%%%%%%%%%%%%%%%%%%%%%%%%%%%%%%%%%%%%%%%%%%%%%%%%%%%%%%%%%%%%%%%%%%%%
%%%%%%%%%%%%%%%%%%%%%%%%%%%%%%%%%%%%%%%%%%%%%%%%%%%%%%%%%%%%%%%%%%%%%%%%%%%%%%%
\begin{frame}[fragile]{插曲:情境}
\begin{itemize}
\item \bftt{equation} 属于一个“\emph{environment}”,也就是排版中插入的特殊情境.
\item 在不同情境下,同一个命令可能会有不同效果。
\begin{exampletwouptiny}
我们可以在行内写
$ \Omega = \sum_{k=1}^{n} \omega_k $
也可以单行显示
\begin{equation}
  \Omega = \sum_{k=1}^{n} \omega_k
\end{equation}
这个公式。
\end{exampletwouptiny}
\vskip 2ex
\item 注意到 $\Sigma$ 在 \bftt{equation} 情境里更大,上下标也换了位置。 
\vskip 1em
{\scriptsize 其实 \bftt{\$...\$} 相当于
\cmdbegin{math}\bftt{...}\cmdend{math}.}
\end{itemize}
\end{frame}

%%%%%%%%%%%%%%%%%%%%%%%%%%%%%%%%%%%%%%%%%%%%%%%%%%%%%%%%%%%%%%%%%%%%%%%%%%%%%%%
%%%%%%%%%%%%%%%%%%%%%%%%%%%%%%%%%%%%%%%%%%%%%%%%%%%%%%%%%%%%%%%%%%%%%%%%%%%%%%%
%%%%%%%%%%%%%%%%%%%%%%%%%%%%%%%%%%%%%%%%%%%%%%%%%%%%%%%%%%%%%%%%%%%%%%%%%%%%%%%
\begin{frame}[fragile]{插曲:情境}
\begin{itemize}
\item \cmdbs{begin} 和 \cmdbs{end} 命令能定义许多情境。
\vskip 2ex

\item \bftt{itemize} 和 \bftt{enumerate} 生成列表
\begin{exampletwouptiny}
\begin{itemize} % 无序列表
\item[-] 符号自定义
\item[什么] 都可以
\end{itemize}

\begin{enumerate} % 有序列表
\item 没有太大自由
\item 但也有选项 % 宏包和自定义命令
\end{enumerate}

\end{exampletwouptiny}
\end{itemize}
\end{frame}

%%%%%%%%%%%%%%%%%%%%%%%%%%%%%%%%%%%%%%%%%%%%%%%%%%%%%%%%%%%%%%%%%%%%%%%%%%%%%%%
%%%%%%%%%%%%%%%%%%%%%%%%%%%%%%%%%%%%%%%%%%%%%%%%%%%%%%%%%%%%%%%%%%%%%%%%%%%%%%%
%%%%%%%%%%%%%%%%%%%%%%%%%%%%%%%%%%%%%%%%%%%%%%%%%%%%%%%%%%%%%%%%%%%%%%%%%%%%%%%
\begin{frame}[fragile]{插曲:宏包}

\begin{itemize}
\item 我们之前用的大多数命令都是
\LaTeX 预设的。

\item \emph{宏包/Packages} 是自定义命令和情境的档案室。

\item 我们使用的宏包需要在头文件里用
\cmdbs{usepackage}命令加载.

\item 例:\emph{American Mathematical Society}的\bftt{amsmath} 宏包。
\begin{minted}[fontsize=\small,frame=single]{latex}
\documentclass{article}
\usepackage{amsmath} % 头文件
\begin{document}
% 这里可以使用 amsmath 的命令了……
\end{document}
\end{minted}
\end{itemize}
\vskip 1em
\begin{center}
\fbox{
\href{https://www.ctan.org/pkg/amsmath?lang=en}{%
\bftt{amsmath}说明文档}}
\end{center}
\end{frame}

%%%%%%%%%%%%%%%%%%%%%%%%%%%%%%%%%%%%%%%%%%%%%%%%%%%%%%%%%%%%%%%%%%%%%%%%%%%%%%%
%%%%%%%%%%%%%%%%%%%%%%%%%%%%%%%%%%%%%%%%%%%%%%%%%%%%%%%%%%%%%%%%%%%%%%%%%%%%%%%
%%%%%%%%%%%%%%%%%%%%%%%%%%%%%%%%%%%%%%%%%%%%%%%%%%%%%%%%%%%%%%%%%%%%%%%%%%%%%%%
\begin{frame}[fragile]{\insertsubsection{}:\bftt{amsmath}示例}
\begin{itemize}
\item 表达式不想编号就用\bftt{equation*} (``加星equation'') 
\begin{exampletwouptiny}
\begin{equation*}
  \Omega = \sum_{k=1}^{n} \omega_k
\end{equation*}
\end{exampletwouptiny}
\item \LaTeX{} 认为数学模式下相邻的字母都是相乘关系,因而都是强调字体。\bftt{amsmath} 里定义了一些常用函数的专有名。
\begin{exampletwouptiny}
\begin{equation*} % 格式不对!
 min_{x,y} (1-x)^2 + 100(y-x^2)^2
\end{equation*}
\begin{equation*} % 赞!
\min_{x,y}{(1-x)^2 + 100(y-x^2)^2}
\end{equation*}
\end{exampletwouptiny}
\item 你也可以用 \cmdbs{operatorname} 自定义专有名称。
\begin{exampletwouptiny}
\begin{equation*}
\beta_i =
\frac{\operatorname{Cov}(R_i, R_m)}
     {\operatorname{Var}(R_m)}
\end{equation*}
\end{exampletwouptiny}
\end{itemize}
\end{frame}

%%%%%%%%%%%%%%%%%%%%%%%%%%%%%%%%%%%%%%%%%%%%%%%%%%%%%%%%%%%%%%%%%%%%%%%%%%%%%%%
%%%%%%%%%%%%%%%%%%%%%%%%%%%%%%%%%%%%%%%%%%%%%%%%%%%%%%%%%%%%%%%%%%%%%%%%%%%%%%%
%%%%%%%%%%%%%%%%%%%%%%%%%%%%%%%%%%%%%%%%%%%%%%%%%%%%%%%%%%%%%%%%%%%%%%%%%%%%%%%
\begin{frame}[fragile]{\insertsubsection{}:\bftt{amsmath}示例}
\begin{itemize}{\small
\item 若需将方程式沿等号对齐
\begin{align*}
(x+1)^3 &= (x+1)(x+1)(x+1) \\
        &= (x+1)(x^2 + 2x + 1) \\
        &= x^3 + 3x^2 + 3x + 1
\end{align*}
可使用 \bftt{align*} 情境。

% for whatever reason, this doesn't play well with the twoup environment
\begin{minted}[fontsize=\small,frame=single]{latex}
\begin{align*}
(x+1)^3 &= (x+1)(x+1)(x+1) \\
        &= (x+1)(x^2 + 2x + 1) \\
        &= x^3 + 3x^2 + 3x + 1
\end{align*}
\end{minted}
\item 用and符号 \keystrokebftt{\&} 分开左栏 (同一行里
$=$前)和右栏($=$后)。
\item 双反斜线 \keystrokebftt{\bs}\keystrokebftt{\bs} 换行}\end{itemize}
\end{frame}


%%%%%%%%%%%%%%%%%%%%%%%%%%%%%%%%%%%%%%%%%%%%%%%%%%%%%%%%%%%%%%%%%%%%%%%%%%%%%%%
%%%%%%%%%%%%%%%%%%%%%%%%%%%%%%%%%%%%%%%%%%%%%%%%%%%%%%%%%%%%%%%%%%%%%%%%%%%%%%%
%%%%%%%%%%%%%%%%%%%%%%%%%%%%%%%%%%%%%%%%%%%%%%%%%%%%%%%%%%%%%%%%%%%%%%%%%%%%%%%
\begin{frame}[fragile]{排版练习 2}

\begin{block}{用 \LaTeX 处理以下文段:}
令 $X_1, X_2, \ldots, X_n$ 为一组期望值 
$\operatorname{E}[X_i] = \mu$,方差 
$\operatorname{Var}[X_i] = \sigma^2 < \infty$
的独立同分布随机变量,且用
\begin{equation*}
S_n = \frac{1}{n}\sum_{i}^{n} X_i
\end{equation*}
表示其平均值。当 $n$ 趋于无穷时, 
随机变量  $\sqrt{n}(S_n - \mu)$ 
将依分布收敛至正态分布 $N(0, \sigma^2)$。
\end{block}
\vskip 2ex
\begin{center}
\fbox{\href{\wlnewdoc{basics-exercise-2.tex}}{%
点击这里在 \wllogo{} 中打开该文档}}
\end{center}
\begin{itemize}
\item 提示:用\cmdbs{infty}输入无穷符号。
\item 试过之后,
\fbox{\href{\wlnewdoc{basics-exercise-2-solution.tex}}{%
这里有我的解答}}.
\end{itemize}
\end{frame}

%%%%%%%%%%%%%%%%%%%%%%%%%%%%%%%%%%%%%%%%%%%%%%%%%%%%%%%%%%%%%%%%%%%%%%%%%%%%%%%
%%%%%%%%%%%%%%%%%%%%%%%%%%%%%%%%%%%%%%%%%%%%%%%%%%%%%%%%%%%%%%%%%%%%%%%%%%%%%%%
%%%%%%%%%%%%%%%%%%%%%%%%%%%%%%%%%%%%%%%%%%%%%%%%%%%%%%%%%%%%%%%%%%%%%%%%%%%%%%%
\begin{frame}{第一节结束}
\begin{itemize}
\item 不错!你已经学会了 \ldots
\begin{itemize}
\item 在 \LaTeX 中排版文档;
\item 使用多种命令;
\item 处理异常情况;
\item 生成漂亮的数学表达式;
\item 插入情境处理特殊文本;
\item 加载宏包。
\end{itemize}
\item 给自己一朵小红花!
\item 第二节里,我们会学习用 \LaTeX{} 写带有章节、引用、图表、引用的多层文档。下节课见!
\end{itemize}
\end{frame}


\end{document}
